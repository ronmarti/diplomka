
%!TEX ROOT=ctutest.tex

\chapter{Dynamický model}

Pro výpočet a predikci spotřeby elektrické energie je potřeba vytvořit matematický dynamický model robota. 
V případě 6-ti osového manipulátoru se jedná o systém se šesti stupni volnosti. K popisu jeho dynamiky je proto potřeba 6 rovnic druhého řádu. Celkově se tedy jedná o systém dvanáctého řádu. 

K odvození těchto rovnic je možné použít jeden ze dvou základních přístupů a to Newton-Eulerovu metodu nebo Euler-Lagrangeovu metodu. 

Newton-Eulerova metoda je založena na přístupu k systému jako k soustavě jednotlivých jeho částí a vyžaduje určení pohybových rovnic každé jednotlivé osy. Protože jsou jednotlivé osy vzájemně kinematicky propojeny, jsou i pohybové rovnice jednotlivých os závislé na pohybu ostatních os. Euler-Lagrangeova metoda naopak přistupuje k systému jako k celku a je založena na určení jeho celkové kinetické a potenciální energie. 

Oba přístupy nakonec vedou ke stejným rovnicím. Protože jsou jednotlivé polohy a dynamika systému popisovány pomocí úhlů na jednotlivých osách, jsou tyto rovnice silně nelineární. V případě robota KR5 se jedná o soustavu 6 rovnic o celkem 24 neznámých (moment, poloha, rychlost a zrychlení pro každou osu). 

Rovnice systému je možné zapsat v následujícím maticovém tvaru jako 

\begin{equation}
T = M(\dot{\theta},\theta)\ddot{\theta} + C(\dot{\theta},\theta)\dot{\theta} + G(\theta) + f_v\dot{\theta} + f_csign(\dot{\theta})
\label{dyn_rovnice_eq}
\end{equation}

kde

\begin{description}
\item[$T = {\big[T_1  \dotsm  T_n\big]}^{T}$] je vektor momentů sil působících na jednotlivé osy robota
\item[$\ddot \theta = {\big[\ddot \theta_1  \dotsm  \ddot \theta_n\big]}^{T}$] je vektor úhlových zrychlení na jednotlivých osách
\item[$\dot \theta = {\big[\dot \theta_1  \dotsm  \dot \theta_n\big]}^{T}$] je vektor úhlových rychlostí na jednotlivých osách
\item[$M\big(\dot \theta, \theta\big)$] je matice setrvačnosti tvořena tenzory setrvačnosti jednotlivých os
\item[$C\big(\dot \theta, \theta\big)$] je matice Coriolisových a odstředivých sil působících na jednotlivé osy
\item[$G\big(\theta\big)$] je matice gravitačních sil působících na jednotlivé osy
\item[$f_v$] je vektor koeficientů viskózního tření v jednotlivých osách
\item[$f_c$] je vektor koeficientů Coulombova suchého tření v jednotlivých osách
\item[$n$] je počet os
\end{description}

K výpočtu okamžité spotřeby elektrické energie je nutné řešit inverzní dynamickou úlohu, kdy se z okamžitých poloh, rychlostí a zrychlení na jednotlivých osách robota vypočítají točivé momenty, kterými působí motory. Moment síly motoru je závislý na proudu protékajícím jeho vinutím. Tuto závislost je často možné aproximovat lineární závislostí a psát jako
\begin{equation}
T\big(t\big) = KI\big(t\big)
\label{torque_current_eq}
\end{equation}

kde

\begin{description}
\item[$T\big(t\big) {\big[Nm\big]}$] moment síly motoru 
\item[$K {\big[Nm/A\big]}$] momentová konstanta 
\item[$I\big(t\big) {\big[A\big]}$] proud protékající motorem 
\end{description}

Momentové konstanty jednotlivých motorů je možné zjistit v jejich dokumentaci. Nástroj TRACE robotu KUKA KR5 takto počítá momenty sil jednotlivých motorů. 

\section{Elektrický výkon}
Poté je již snadné z momentů na motorech určit jejich proudy a tím i elektrický výkon.

\section{Solver ReDySim}

Pro usnadnění odvození soustavy rovnic pro robota o 6 stupních volnosti a pro případné rozšíení použití i na jiný typy robotů byl použit skript pro matematický nástroj MATLAB využívající solver ReDySim Symbolic\cite{redysim}. Tento nástroj byl vyvinut na univerzitě v Dillí a je bezplatně k dispozici ke stažení a použití v MATLABu. Je schopen generovat rovnice pro libovolný počet os. Vstupními parametry jsou DH-parametry robota a fyzické parametry s numerickými nebo symbolickými hodnotami. 


