
%!TEX ROOT=ctutest.tex

\chapter{Úvod}


Cílem této práce je vytvoření matematického modelu průmyslového robotického manipulátoru za účelem modelování jeho spotřeby elektrické energie při výkonu daných robotických operací. Tato data by poté měla sloužit pro další identifikaci.

Modelování průmyslových robotů již bylo předmětem mnoha prací a projektů a to už od jejich návrhu a prvních použití. Nejčastěji tyto modely slouží pro návrh řízení robota nebo jeho optimalizaci. Tato práce je zaměřena na modelování robota a identifikaci z hlediska jeho spotřeby elektrické energie. Protože dynamické parametry robota často nejsou známy, je nejprve nutné tyto parametry identifikovat.

Identifikací dynamických parametrů se již zabývalo několik prací. V článcích \cite{par_iden_rob}\cite{clos_dyn_par} je použita identifikace dynamických parametrů metodou nejmenších čtverců. Práce \cite{dyn_mod_ind}\cite{dyn_ind_mits} se zabývají identifikací touto metodou robota Mitsubishi PA-10. Jiným způsobem se postupuje v článcích \cite{dyn_ind_man} a \cite{akeel}, které se zabývají identifikací systému pomocí 3D modelu. V této práci je k identifikaci je použita metoda identifikace pomocí nejmenších čtverců.

Průmyslové robotické manipulátory jsou dnes již nedílnou součástí průmyslové sféry. Na rozdíl od jednoduchých jednoúčelových průmyslových strojů, které jsou úzce specializované jen na jeden typ operace, jsou průmyslové roboty víceúčelové a jsou schopny vykovávat téměř libovolnou operaci. Jsou omezeny jen vlastní geometrií, uspořádáním pracovního prostoru ve které se provozují a mechanickými vlastnostmi aktuátorů a jednotlivých prvků robota. Díky těmto vlastnostem je jeden průmyslový robot schopen vykovávat operace, ke kterým by jinak bylo potřeba více strojů, a to jen změnou programu.

Dnes se roboty v průmyslu používají pro mnoho typů operací. Patří mezi ně svařování, montáž, manipulace s materiálem, lakování, vrtání a mnoho dalších.







