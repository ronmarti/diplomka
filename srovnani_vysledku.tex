\chapter{Srovnání výsledků}

Odvozený a identifikovaný model z kapitoly \ref{identifikovane_parametry_ch}, který dává do vztahu točivé momenty s polohami, úhlovými rychlostmi a úhlovými zrychleními na jednotlivých osách (rovnice \ref{celkova_dyn_rovnice_eq}) je možné použít k výpočtu celkového elektrického výkonu robota (viz sekce \ref{el_vykon_ch}).

Vypočítané momenty sil na jednotlivých osách se pomocí momentových konstant převedly na efektivní hodnoty proudů protékajících vinutími motorů. Nahrazením vinutí motorů obvodem s odporem a indukčností zapojenými v sérii byly z těchto hodnot proudů vypočítány jednotlivé hodnoty efektivního napětí na svorkách motorů. Vynásobením hodnot napětí a proudů byly vypočítány elektrické výkony na jednotlivých osách. Celkový elektrický výkon robota je poté dán součtem všech dílčích výkonů na všech osách.  