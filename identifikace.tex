
%!TEX ROOT=ctutest.tex

\chapter{Identifikace systému}

U robotického manipulátoru zpravidla nejsou zcela známy informace o dynamických parametrech robota, jako jsou momenty setrvačnosti, hmotnosti nebo koeficienty tření jednotlivých os. Tyto informace nejsou v běžných situacích poskytovány ani samotnými výrobci robotů. Je to hlavně proto, že pro zákazníka nejsou tyto údaje důležité, protože se robotické manipulátory dodávají jako hotové uzavřené systémy připravené k použití. Jejich řízení je již implementováno v řídícím systému robota.

Z toho důvodu je nutné tyto parametry nějakým způsobem odvodit. Toho je možné docílit několika hlavními způsoby.

\section{Z přímého měření součástí robota}

Dynamické parametry je možné určit rozebráním robota na menší součásti a přímým měřením jejich dynamických vlastností. Tento způsob se jeví jako nejpřirozenější.

Určení parametrů takovýmto způsobem je ale možné pouze u jednoduchých laboratorních modelů robota tvořených malým počtem součástí. U větších a složitějších robotů je tento způsob náročný časově i způsobem provedení. Jednotlivé linky sestávají z více komponent, jako jsou převodovky motorů, napájecí a komunikační vedení motorů atd. Ty dále sestávají z dalších komponent.

Další nevýhodou je nemožnost zobecnění tohoto způsobu na více typů robotů. Každý typ robota by se musel rozebrat a změřit, i kdyby se jednalo o robota podobné konstrukce.
Proto se tato práce tímto postupem dále nezabývá.    

\section{Z 3D modelu}

Výrobce poskytuje k robotu KUKA KR5 3D model. Ten je možné analyzovat v nástrojích CAD jako je například AutoCAD nebo Siemens NX, které jsou schopny počítat momenty setrvačnosti a hmotnosti libovolně složitých objektů. Výhodou tohoto postupu je jeho rychlost a jednoduchost. Navíc je takto možné získat hledané parametry bez nutnosti přístupu k opravdovému fyzickému robotu. Je také možné tento postup zobecnit na libovolný typ robota.
   
3D model ale popisuje pouze povrchovou geometrii jednotlivých komponent robota a neobsahuje informace o jejich vnitřní konstrukci ani hustotě použitých materiálů. Je sice možné považovat jednotlivá ramena robota za homogenní a hmotnost
odhadnout z celkové hmotnosti robota udávané v datasheetu, tento postup ale dává jen velmi hrubý odhad dynamických parametrů. Navíc z 3D modelu není možné získat informace o koeficientech tření os. 
Tento postup je zde použit pouze pro účely porovnání určených hodnot.

\section{Z rovnic}

Přestože jsou dynamické rovnice robota \ref{dyn_rovnice_eq} nelineární vůči zobecněným souřadnicím, jsou lineární vůči jednotlivým složkám dynamických parametrů. Proto je možné je přepsat do tvaru
\begin{equation}
T = H(\ddot{\theta},\dot{\theta},\theta)P
\label{eq_lin_par}
\end{equation}
kde
\begin{description}
\item[$T = {\big[T_1  \dotsm  T_n\big]}^{T}$] je vektor momentů 
\item[$P = {\big[P_1  \dotsm  P_n\big]}^{T}$] je vektor neznámých dyn. parametrů jednotlivých os
\end{description} \noindent
a \ \ \ \ \ \ $P_i = {\big[I_{ixx} \ I_{ixy} \ I_{iyy} \ I_{iyz} \ I_{izz} \ I_{izx} \ m_ir_{ix} \ m_ir_{iy} \ m_ir_{iz} \ m_i f_{vi} \ f_{ci}\big]}^{T}$ \\
\\
\\
kde
\noindent
\begin{description}
\item[$I_{ijk}$] je složka setrvačnosti pro link $i$ vůči souřadnicím $j$ a $k$
\item[$r_{ij}$] je složka vektoru těžiště linku $i$ vyjádřená v souřadnici $x$
\item[$m_{i}$] je hmotnost linku $i$
\item[$f_{vi}$] je koeficient viskózního tření linku $i$
\item[$f_{ci}$] je koeficient Coulombova tření linku $i$
\end{description}

Počet neznámých je možné zredukovat, protože některé parametry dynamiku robota neovlivní. Je to způsobeno tím, že se některé linky mohou otáčet jen kolem některé z os. Příkladem může být osa 1 (spojená se zemí), která se v prostoru může otáčet jen kolem jedné osy. Zároveň je možné si model zjednodušit uvažováním pouze prvků na hlavní diagonále tenzorů setrvačnost a zanedbáním prvků mimo ni.

V následující tabulce (tabulka \ref{tab_hled_param}) je přehled hledaných neznámých dynamických parametrů.   
\\

\begin{table}[htbp]
  \centering
  \caption{Tabulka nezámých parametrů}
    \begin{tabular}{c|lllllllll}
    \multicolumn{1}{c|}{Osa} & \multicolumn{9}{c}{Neznámé parametry}  \\
    \hline
    1 &       	  &	          &           &          &          &          & $I_{1z}$& $f_{v1}$ & $f_{c1}$ \\
    2 & $I_{2xx}$ & $I_{2yy}$ & $I_{2zz}$ & $d_{2x}$ & $d_{2y}$ & $d_{2z}$ & $m_{2}$ & $f_{v2}$ & $f_{c2}$ \\
    3 & $I_{3xx}$ & $I_{3yy}$ & $I_{3zz}$ & $d_{3x}$ & $d_{3y}$ & $d_{3z}$ & $m_{3}$ & $f_{v3}$ & $f_{c3}$ \\
    4 & $I_{4xx}$ & $I_{4yy}$ & $I_{4zz}$ & $d_{4x}$ & $d_{4y}$ & $d_{4z}$ & $m_{4}$ & $f_{v4}$ & $f_{c4}$ \\
    5 & $I_{5xx}$ & $I_{5yy}$ & $I_{5zz}$ & $d_{5x}$ & $d_{5y}$ & $d_{5z}$ & $m_{5}$ & $f_{v5}$ & $f_{c5}$ \\
    6 & $I_{6xx}$ & $I_{6yy}$ & $I_{6zz}$ & $d_{6x}$ & $d_{6y}$ & $d_{6z}$ & $m_{6}$ & $f_{v6}$ & $f_{c6}$ \\
    \end{tabular}%
  \label{tab_hled_param}%
\end{table}%


Naměřením průběhů momentů, poloh, rychlostí a zrychlení na jednotlivých osách a jejich dosazením do lineární rovnice \ref{eq_lin_par} lze pak tuto rovnici řešit ve tvaru 
\begin{equation}
P = H(\ddot{\theta},\dot{\theta},\theta)^{-1}T
\label{eq_lin_par_inv}
\end{equation}
Důležité je na trajektorii mít tolik bodů, aby z rovnice \ref{eq_lin_par} vznikla rovnice přeurčená. Takovou rovnici je poté možné řešit například použitím metody nejmenších čtverců. Ta minimalizuje střední odchylku mezi skutečnými a odhadnutými parametry. 

\section{Excitační trajektorie}

Aby bylo možné robota takto identifikovat, je potřeba s robotem provést takové pohyby, aby byly vybuzeny všechny dynamické složky robota, tzn. aby se projevily všechny neznámé parametry. Ve vědeckých článcích a v jiných publikacích např. \cite{clos_dyn_par}\cite{dyn_mod_ind}\cite{dyn_ind_mits} se na jednotlivých osách doporučují průběhy, které je možné popsat konečnou Fourierovou řadou. Jejich výhodou je, že díky vlastnostem harmonické funkce jsou poté jak polohy, tak i rychlosti a zrychlení rovněž kombinací harmonických průběhů a tím se minimalizuje vliv šumu měření. 

Protože se robot používá převážně pro polohování, jeho řídící systém zpravidla neumožňuje na osách provádět čistě harmonické průběhy. Řídící systém robota KUKA KR5 umožňuje pouze nastavit požadované koncové polohy os a rychlosti/zrychlení, s jakými jich má robot dosáhnout. Z toho důvodu je nutné robotu poskytnout sérii bodů popisujících harmonický průběh. Výsledná trajektorie robota poté bude pouze aproximací harmonického průběhu.  

Bohužel takové průběhy momentálně nejsou k dispozici a kvůli odstavení robota je v tuto chvíli nelze naměřit. Identifikace tedy byla zatím provedena na dříve naměřených průbězích, kdy robot postupně 10x zahýbal osou 1, poté se zastavil, zahýbal osou 2 a tak pokračoval až k poslední ose. 

\section{Postup identifikace}

Při identifikaci parametrů se postupovalo od poslední, šesté osy (konečného linku) k první. Nejprve se pevně zafixovaly ostatní osy a z průběhů na šesté ose se metodou nejmenších čtverců pomocí rovnice \ref{eq_lin_par_inv} určily její dynamické parametry. Poté se tento postup zopakoval pro předchozí osu až k ose první.   

Takto se podařilo odvodit některé dynamické parametry. Protože se ale jednalo o šest nezávislých měření pro šest pohybů s ostatními osami pevně zafixovanými, nepokryla se kompletní škála pohybů a neprojevila se při těchto průbězích veškerá dynamika. Proto se nepodařilo odvodit všechny neznámé parametry. 

