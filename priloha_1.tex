\chapter{Návod k aplikaci MongoDB data exporter}
\label{appendix1}
\section{Kompilace}

Aplikace MongoDB data exporter je napsána v jazyce C. Zdrojový kód aplikace je součástí této práce a je přiložen v souboru ... .

Aplikace je vytvořena jako multiplatformní. Zdrojový kód obsahuje pouze knihovny, které je možné použít na libovolné platformě. Pro vytvoření spouštěče aplikace je nezbytné zdrojové soubory zkompilovat pro použití na dané požadované platformě.

Pro kompilaci pro systém Linux je možné použít vestavěnou sadu kompilátorů GNU Compiler Collection použitím příkazu \texttt{gcc}. Kompilaci pro MS Windows je možné provést například pomocí nástroje Visual Studio příkazem \texttt{cl.exe}. Dynamicky linkované knihovny DLL jsou přiloženy ke zdrojovému kódu.  

\section{Použití}

Aplikace je používána k exportu dat z databáze MongoDB měření elektrického výkonu průmyslových manipulátoru v robotické lince.

Umožňuje uživateli správu záloh databáze (mazání starých záloh, vytvoření nových), export dat ze specifikované databáze a kolekce do určeného souboru a nakonec její čištění.

Aplikaci je možné použít jako vícevláknovou. Uživatel má možnost si zvolit počet vláken použitých pro převod a export dat za účelem zrychlení zpracování velkého množství dat. Každé vlákno poté zpracovává svůj vlastní úsek dat u ukládá jej do samostatného souboru který je označován jako pn (n ... počet vláken). Aplikace podporuje až 16 současně spuštěných vláken. 

Uživatel má dále možnost specifikovat časový úsek, ze kterého chce data exportovat. Prosím vezměte na vědomí, že případě velkého množství dat může určení hranic pro export dat trvat delší dobu. Požadované hranice časového úseku jsou poté přidány do názvu výstupního souboru, aby z názvu souboru bylo možné poznat, jaká data se v něm nachází.

Aplikace se spouští spuštěním souboru \texttt{cl.exe}. Tento soubor je možné spouštět samostatně nebo s následujícími parametry.

Vstupní parametry:

\setlength{\leftskip}{1cm}

\noindent
-h \hspace{0.5cm} - \hspace{0.5cm} Zobrazí nápovědu\newline
-c \hspace{0.5cm} - \hspace{0.5cm} Specifikuje umístění konfiguračnoho soubou

\hspace{0.5cm}Příklad: \hspace{0.5cm} \texttt{run -c C:/files/data/config.conf}

\hspace{0.5cm}přečte konfigurační soubor \texttt{config.conf} umístěný v \texttt{C:/files/data/}

\setlength{\leftskip}{2cm}

V případě, že není uživatelem poskytnuto umístění konfiguračního souboru, je použito výchozí nastavení - \texttt{config.conf} v adresáři souboru \texttt{run}

\setlength{\leftskip}{1cm}
\noindent
-a \hspace{0.5cm} - \hspace{0.5cm} Specifikuje odpovědi na otázky vyskytující se v programu

\setlength{\leftskip}{2cm}
\noindent
Povolené symboly: Y (ano) nebo n (ne)
\newline\noindent
Otázky:

\setlength{\leftskip}{3cm}
\noindent
1. Smazat staré databáze? (Y,n)\newline
2. Vytvořit zálohu vybrané databáze? (Y,n)\newline
3. Exportovat data? (Y,n)\newline
4. Vyčistit vybranou databázi? (Y,n)\newline

\setlength{\leftskip}{2cm}
\noindent
Příklad: \hspace{0.5cm} \texttt{run -a YnYn}

\noindent    
Aplikace spuštěná s těmito parametry smaže staré záložní databáze, nevytvoří nové zálohy, exportuje data a nevyčistí vybranou databázi           

\noindent
Pokud nejsou parametry specifikovány žádné odpovědi, aplikace se spustí normálně a zeptá se na tyto otázky během jejího běhu

\setlength{\leftskip}{0cm}
\noindent
Formát konfiguračního souboru:

\setlength{\leftskip}{1cm}
\noindent
\texttt{URI=mongodb://localhost:27017} - adresa databáze MongoDB \newline
\texttt{DB\_NAME=database} - jméno MongoDB databáze k připojení \newline
\texttt{COLL\_NAME=collection} - jméno kolekce k připojení \newline
\texttt{OUT\_FILE=C:/data\_export.csv} - umístění a název výstupního souboru pro export dat \newline
\texttt{NUM\_THREADS=2} - počet vláken/výstupních souborů pro export dat \newline
\texttt{DATA\_FROM=0} - čas od kterého mají být exportována data \newline
\texttt{DATA\_TO=1701041215} - čas do kterého mají být exportována data \newline

\setlength{\leftskip}{1.5cm}
\noindent
Formát časových hranic: \hspace{0.2cm} RRMMDDhhmm RR-rok, MM-měsíc, DD-den, hh-hodina, mm-minuta                     

\noindent
Pokud nechcete použít časový úsek, zadejte hodnoty 0

\setlength{\leftskip}{1cm}
\noindent 
Aplikace spuštěná s tímto konfiguračním souborem se připojí k databázi MongoDB běžící na místní adrese (localhost) na portu 27010. Dále se pokusí připojit k databázi \texttt{database} a ke kolekci \texttt{collection}. Data budou exportována od začátku až do 4.1.2017 - 12:15. Protože jsou vybrána dvě vlákna, budou vytvořeny 2 výstupní soubory. Data budou exportována do těchto dvou výstupních souborů:

\setlength{\leftskip}{2cm}
\noindent 
\texttt{data\_export\_0-1701041215\_p1.csv} \newline
\texttt{data\_export\_0-1701041215\_p2.csv} \newline
umístěných v \texttt{C:/}