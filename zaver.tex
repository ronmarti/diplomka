\chapter{Závěr}

V této práci byly prostudovány způsoby vytvoření modelu šestiosého průmyslového robotu a identifikace jeho dynamických parametrů. Důraz byl kladen na vytvoření metodiky identifikace, kterou by bylo možné použít na široké spektrum typů robotu. 

Tento model by měl sloužit pro modelování spotřeby elektrické energie robotického systému. Z modelu je možné z pohybu při požadovaných robotických operacích predikovat, jaké množství energie tato operace spotřebuje a díky tomu daný proces optimalizovat.

Z hlediska standardizace identifikace dynamických parametrů byly prostudovány a vzájemně porovnány dva postupy, identifikace z 3D modelu robotu a z jeho rovnic. Byly popsány postupy obou způsobů identifikace a analyzovány jejich výhody a nevýhody. 

Dynamické parametry byly také analyzovány z hlediska jejich vlivu na přesnost dynamického modelu robotu. Díky tomu je možné určit, které dynamické parametry mají hrají v přesnosti modelu největší roli a tedy, které parametry je vhodné identifikovat s co nejvyšší přesností.

Pro identifikaci dynamických parametrů z rovnic robotu byl vytvořen skript pro MATLAB, který je schopen vytvořit matematický model, identifikovat jeho parametry, simulovat výsledky a porovnat je s měřením. Skript také umožňuje provádět analýzu vlivu odchylek odhadnutých parametrů na přesnost modelu. 

Data získaná pomocí odvozeného modelu byla porovnána se skutečným reálným měřením. Dále byla provedena analýza důvodů odchylek mezi predikcí a měřením a navrženy způsoby, jak je eliminovat.  

V poslední řadě byla vytvořena aplikace sloužící k exportu dat z databáze dlouhodobého měření energetické spotřeby robotické buňky. Aplikace je schopna provádět zálohu, čištění a export dat z databáze. Byla vytvořena pro zjednodušení a zefektivnění přípravy dat z databáze měření energetické spotřeby robotické buňky používané v závodu společnosti ŠKODA AUTO a.s. v Kvasinách. 

\section{Práce do budoucna}

Za účelem snížení odchylky mezi daty predikovanými pomocí vytvořeného matematického modelu a daty naměřenými způsobem popsaným v této práci je potřeba provést některé úpravy, započítávající vliv frekvenčních měničů a dalších elektrických součástí v rozvaděči robotického systému. 

Jednou z možných úprav je doplnění do modelu robotu rovnic popisujících elektrickou část frekvenčních měničů řídících motory robotu, stabilizátorů napájení a dalších elektrických prvků v elektrickém systému robotu.

Druhou možností je úprava způsobu měření reálné spotřeby robotu tak, aby energie těchto elektrických zařízení nebyla v měření započítána.